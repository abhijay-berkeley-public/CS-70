\documentclass[11pt, notitlepage]{report}

	\usepackage[margin=1in]{geometry}
	\usepackage{amsmath,amsthm,amssymb,amsfonts}
	\usepackage{enumitem}
	\usepackage{systeme}

	\newcommand{\N}{\mathbb{N}}
	\newcommand{\Z}{\mathbb{Z}}
	\newcommand{\R}{\mathbb{R}}
	\newcommand{\A}{\alpha}
	\newcommand{\ora}[1]{\overrightarrow{#1}}
	\newcommand{\Question}[1]{\newpage\section{#1}}
	\usepackage[parfill]{parskip}
	\usepackage{mathtools}
	\newenvironment{solution}{\paragraph{Solution:}}{\hfill}
	\newenvironment{theorem}{\paragraph{Theorem:}}{\hfill}
	\newenvironment{subtheorem}[1]{\paragraph{\small Subtheorem #1:}}{\hfill}
	\newenvironment{definition}{\paragraph{Definition:}}{\hfill}
	\newenvironment{problem}[2][Problem]{\begin{trivlist}
	\item[\hskip \labelsep {\bfseries #1}\hskip \labelsep {\bfseries #2.}]}{\end{trivlist}}
	
	\usepackage{pgfplots}
	\usetikzlibrary{arrows}
	\usetikzlibrary{decorations.markings}
	\usetikzlibrary{datavisualization}
	\usetikzlibrary{datavisualization.formats.functions}
	%\usepackage{pstricks-add}

	\pgfplotsset{every axis/.append style={
	                   axis x line=middle,    % put the x axis in the middle
	                   axis y line=middle,    % put the y axis in the middle
	                   axis line style={<->,color=gray}, % arrows on the axis
	                   xlabel={$x$},          % default put x on x-axis
	                   ylabel={$y$},          % default put y on y-axis
	           }}
	\pgfplotsset{compat=1.15}
	
	\newcommand{\pgraph}[4]{
		\begin{center}
		
		\begin{tikzpicture}
		\begin{axis}[
		   trig format plots=rad,
		   axis equal,
		   grid=both
		]
		\addplot [domain=#3:#4, variable=\t, samples=50, black, decoration={
		   markings,
		   mark=between positions 0.2 and 1 step 4em with {\arrow [scale=1.5]{stealth}}
		   }, postaction=decorate]
		({#1}, {#2});
		
		\end{axis}
		\end{tikzpicture}
		
		\end{center}
	}


   \newcommand{\cgraph}[3]{
	   \begin{center}
	
	   \begin{tikzpicture}
	   \begin{axis}[
	       trig format plots=rad,
	       axis equal,
	       grid=both
	   ]
	   \addplot [domain=#2:#3, variable=\x, samples=50, black, decoration={
	       markings,
	       mark=between positions 0.2 and 1 step 4em with {\arrow [scale=1.5]{stealth}}
	       }, postaction=decorate]
	   {#1};
	
	   \end{axis}
	   \end{tikzpicture}
	
	   \end{center}
	}


	
	\makeatletter
	\newcommand*{\toccontents}{\@starttoc{toc}}
	\makeatother


\begin{document}
   \title{CS70: Homework \#5}
   \author{Abhijay Bhatnagar}
   \maketitle
   \toccontents



\setcounter{secnumdepth}{0} %% no numbering

\Question{Problem 1: Quick Computes}
Simplify each expression using Fermat's Little Theorem. 
\begin{enumerate}[label=(\alph*)]
  \item $3^{33} \pmod{11}$
	\begin{solution} $5\pmod{11}$.
	\begin{align*}
		3^{33} \pmod{11}&\equiv (3^3)^{11} \pmod{11} \\
		&\equiv 27^{11} \pmod{11} \\
		&\equiv 27\pmod{11} \\
		&\equiv 5\pmod{11}
	\end{align*}
	\end{solution}
  \item $10001^{10001} \pmod{17}$
	\begin{solution} 5 (mod 17).
	\begin{align*}
		10001^{10001} \pmod{17} &\equiv 10001^{10000 + 1} \pmod{17} \\
		&\equiv (10001^{625^{17-1}})\cdot 10001 \pmod{17} \\
		&\equiv (1)\cdot 10001 \pmod{17} \\
		&\equiv 5 \pmod{17} 
	\end{align*}
		
	\end{solution}
  \item $10^{10} + 20^{20} + 30^{30} + 40^{40} \pmod{7}$
  \begin{solution}1 (mod 7).
  	\begin{align*}
  		10^{10} + 20^{20} + 30^{30} + 40^{40} \pmod{7} &\equiv 3^{10} + 6^{20} + 2^{30} + 5^{40} \pmod{7} \\
  		&\equiv 3^{4} + 6^{2} + 2^{0} + 5^{4} \pmod{7} \\
  		&\equiv 9^2 + 36 + 1 + 25^{2} \pmod{7} \\
  		&\equiv 4 + 1 + 1 + 4^{2} \pmod{7} \\
  		&\equiv 22 \pmod{7} \\
  		&\equiv 1 \pmod{7} \\
  	\end{align*}
  \end{solution}
\end{enumerate}

\Question{Problem 2: RSA Practice}

Bob would like to receive encrypted messages from Alice via RSA.
\begin{enumerate}[label=(\alph*)]
    \item Bob chooses $p = 7$ and $q = 11$. His public key is $(N,e)$.
    What is $N$?
	\begin{solution}
		N=77
	\end{solution}
    \item What number is $e$ relatively prime to?
	\begin{solution}
		$e$ is relatively prime to (7-1)(11-1)= 60.
	\end{solution}
    \item $e$ need not be prime itself, but what is the smallest prime number $e$ can be? Use this value for $e$ in all subsequent computations.
	\begin{solution}
		Smallest value for $e=7.$
	\end{solution}
    \item What is $\gcd(e,(p-1)(q-1))$?
	\begin{solution}
		$\gcd(e,(p-1)(q-1))=1$
	\end{solution}
    \item What is the decryption exponent $d$? 
    \begin{solution} 
    \begin{align*}
			d&=e^{-1} \pmod{60}\\
			&=7^{-1} \pmod{60}\\
			&=-17 \pmod{60}\\
			&=43 \pmod{60}\\
		\end{align*}
	\end{solution}
    \item Now imagine that Alice wants to send Bob the message $30$. She applies her encryption function $E$ to $30$. What is her encrypted message?
	\begin{solution}
		$E(30)=30^{7}\pmod{77}=2\pmod{77}$ (Using extended Euclid's.)
	\end{solution}
    \item Bob receives the encrypted message, and applies his decryption function $D$ to it.
    What is $D$ applied to the received message?
    \begin{solution}
    	$D(2)=2^{43}\pmod{77}=30\pmod{77}$
    \end{solution}

\end{enumerate}


\Question{Problem 3: Squared RSA}

\begin{enumerate}[label=(\alph*)]
  \item Prove the identity $a^{p(p - 1)} \equiv 1 \pmod{p^2}$, where $a$ is coprime to $p$, and $p$ is prime. (Hint: Try to mimic the proof of Fermat's Little Theorem from the notes.)
	\begin{solution}
		We have to show $(\forall p\in \text{primes},\forall a : gcd(p,a)=1)(a^{p(p - 1)} \equiv 1 \pmod{p^2})$.
		\begin{proof}
			Let $S$ denote the set of nonzero integers which have an inverse mod $p^2$, i.e. \[S=\{1,2,...,p^2-1\} / \{p, 2p , ...,(p-1)p\} \] Note: S will have exactly $(p^2-1) - (p-1) = p^2-p$ terms. Now, consider the sequence \[S'= a^p,2a^p,...,(p^2-1)a^p \pmod{p^2}\]. Since $a$ and $p$ are coprime, we know that $a^p$ and $p^2$ must also be coprime (because greatest common prime factor of $a$ and $p$ is 1, and that doesn't change in $a^p$ and $p^2$, and all other factors have unique prime factorization). However, each term with a coefficient that is a scalar multiple of $p$ is not coprime, so let us redefine $S'$ to exclude those elements. \[S'= \{a^p,2a^p,...,(p^2-1)a^p\}/\{pa^p,2pa^p,...,(p-1)a^p\} \pmod{p^2}\]
			
			From here, we know that every element in $S'$ must be distinct. None of them are zero, and there are exactly $(p^2-1) - (p-1) = p^2-p$ terms $\implies S$ and $S'$ contain the same elements mod $p^2$.
			
			Now if both sets contain the same elements mod $p^2$, then the products must also be equivalent mod $p^2$.
			
			\[\implies a^{(p^2-p)} \times \prod_{n\in S}{n} \equiv \prod_{n\in S}{n} \pmod{p^2} \]
			Since each $n\in S$ is coprime to $p^2$, it must have an inverse mod $p^2$. Let us call this inverse $S^{-1}$. We can multiply both sides by $S^{-1}$ to remove the product term.
			\[\implies a^{(p^2-p)} \times \prod_{n\in S}{n} \times S^{-1} \equiv \prod_{n\in S}{n} \times S^{-1} \pmod{p^2} \]
			\[\implies a^{p(p-1)} \equiv 1 \pmod{p^2} \]

		\end{proof}
	\end{solution}
	  \newpage

  \item Now consider the RSA scheme: the public key is $(N = p^2 q^2, e)$ for primes $p$ and $q$, with $e$ relatively prime to $p(p-1)q(q-1)$. The private key is $d = e^{-1} \pmod{p(p-1)q(q-1)}$.
  Prove that the scheme is correct for $x$ relatively prime to both $p$ and $q$, i.e.\ $x^{ed} \equiv x \pmod{N}$.
 
  \begin{solution}
  	To prove the statement, we have to prove: \begin{equation}\forall x\text{ relatively prime to $p$ and $q$}, (x^e)^d=x\pmod{N}\end{equation}
  \begin{proof}
  	Consider the exponent, $ed$. By the definition of $d$, we know that $ed= 1 \pmod{p(p-1)q(q-1)}$; hence we can write $ed= 1 + kp(p-1)q(q-1)$, therefore:
  	\begin{equation}x^{ed}-x=x^{1 + kp(p-1)q(q-1)}-x=x(x^{kp(p-1)q(q-1)}-1)\end{equation}
  	From (1), we want to show that this last expression is 0 mod N. We claim this is so because $x(x^{kp(p-1)q(q-1)}-1)$ is divisible by $p^2$. To show this, let us consider 2 cases:
  	\begin{enumerate}[label=\arabic*.)]
  	\item $x$ is not a multiple of $p^2$. In this case, since $x\not=0\pmod{p^2}$, we can use our results from (a) to deduce that $x^{kp(p-1)q(q-1)}-1=0 \pmod{p^2}$, as desired.
  	\item $x$ is a multiple of $p^2$. In this case, since $x\not=0\pmod{p^2}$ is a multiple of x, it is trivially divisible by $p^2$.
  	\end{enumerate}
  	By a symmetrical argument, the expression is also divisible by $q^2$, therefore it is divisible by both $p^2$ and $q^2$, and must also be divisible by their products, $p^2q^2=N$. This implies the final expression in (2) $=0\pmod{N}$, which was the necessary condition for our proof.
  \end{proof}
  \end{solution}
  
  \item Prove that this scheme is at least as hard to break as normal RSA; that is, prove that if this scheme can be broken, normal RSA can be as well. We consider RSA to be broken if knowing $pq$ allows you to deduce $(p - 1)(q - 1)$. We consider squared RSA to be broken if knowing $p^2q^2$ allows you to deduce $p(p - 1)q(q - 1)$.
 \begin{solution}
  	The difficult of breaking RSA comes down to difficult of factoring, which is generally accepted to be \textit{hard}. The difficult of breaking squared RSA is a question of figuring out the nontrivial factors of $N=p^2q^2$, which are \{$p,p,q,q$\} (and combinations of those), which are also the prime factors of the normal RSA $N$. If we are able to find out the factors of squared RSA, it means we have deduced we have also found the factors of the normal RSA. Additionally, it means we've found $p(p - 1)q(q - 1)$. From there, we can easily (read: low complexity) divide by the $\sqrt{p^2q^2}=pq$ to get $(p - 1)(q - 1)$, which is solution to cracking the conventional RSA. 
  \end{solution}
\end{enumerate}

\end{document}


