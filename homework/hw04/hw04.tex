\documentclass[11pt, notitlepage]{report}

	\usepackage[margin=1in]{geometry}
	\usepackage{amsmath,amsthm,amssymb,amsfonts}
	\usepackage{enumitem}
	\usepackage{systeme}

	\newcommand{\N}{\mathbb{N}}
	\newcommand{\Z}{\mathbb{Z}}
	\newcommand{\R}{\mathbb{R}}
	\newcommand{\A}{\alpha}
	\newcommand{\ora}[1]{\overrightarrow{#1}}
	\newcommand{\Question}[1]{\newpage\section{#1}}
	\usepackage[parfill]{parskip}
	\usepackage{mathtools}
	\newenvironment{solution}{\paragraph{Solution:}}{\hfill}
	\newenvironment{theorem}{\paragraph{Theorem:}}{\hfill}
	\newenvironment{subtheorem}[1]{\paragraph{\small Subtheorem #1:}}{\hfill}
	\newenvironment{definition}{\paragraph{Definition:}}{\hfill}
	\newenvironment{problem}[2][Problem]{\begin{trivlist}
	\item[\hskip \labelsep {\bfseries #1}\hskip \labelsep {\bfseries #2.}]}{\end{trivlist}}
	
	\usepackage{pgfplots}
	\usetikzlibrary{arrows}
	\usetikzlibrary{decorations.markings}
	\usetikzlibrary{datavisualization}
	\usetikzlibrary{datavisualization.formats.functions}
	%\usepackage{pstricks-add}

	\pgfplotsset{every axis/.append style={
	                   axis x line=middle,    % put the x axis in the middle
	                   axis y line=middle,    % put the y axis in the middle
	                   axis line style={<->,color=gray}, % arrows on the axis
	                   xlabel={$x$},          % default put x on x-axis
	                   ylabel={$y$},          % default put y on y-axis
	           }}
	\pgfplotsset{compat=1.15}
	
	\newcommand{\pgraph}[4]{
		\begin{center}
		
		\begin{tikzpicture}
		\begin{axis}[
		   trig format plots=rad,
		   axis equal,
		   grid=both
		]
		\addplot [domain=#3:#4, variable=\t, samples=50, black, decoration={
		   markings,
		   mark=between positions 0.2 and 1 step 4em with {\arrow [scale=1.5]{stealth}}
		   }, postaction=decorate]
		({#1}, {#2});
		
		\end{axis}
		\end{tikzpicture}
		
		\end{center}
	}


   \newcommand{\cgraph}[3]{
	   \begin{center}
	
	   \begin{tikzpicture}
	   \begin{axis}[
	       trig format plots=rad,
	       axis equal,
	       grid=both
	   ]
	   \addplot [domain=#2:#3, variable=\x, samples=50, black, decoration={
	       markings,
	       mark=between positions 0.2 and 1 step 4em with {\arrow [scale=1.5]{stealth}}
	       }, postaction=decorate]
	   {#1};
	
	   \end{axis}
	   \end{tikzpicture}
	
	   \end{center}
	}


	
	\makeatletter
	\newcommand*{\toccontents}{\@starttoc{toc}}
	\makeatother


\begin{document}
   \title{CS 70: Homework \#4}
   \author{Abhijay Bhatnagar}
   \maketitle
   \toccontents



\setcounter{secnumdepth}{0} %% no numbering
\Question {Problem 1: Modular Arithmetic Solutions}

Find all solutions (modulo the corresponding modulus) to the following equations.  Prove that there are no other solutions (in a modular setting) to each equation.

\begin{enumerate}[label=(\alph*)] 
\item $2x \equiv 5 \pmod{15}$
\begin{solution}
	10, which is unique modulo 15.
	\begin{proof}
		The inverse of 2 (mod 15) is 8 ($2\cdot 8=16=1$ mod 15). From there, we solve the original equation:
		\begin{align*}
			2\cdot 2^{-1} \text{ (mod 15)} &= 5\cdot 2^{-1} \text{ (mod 15)} \\
			&= 5 \cdot 8 \text{ (mod 15)} \\
			&= 40 \text{ (mod 15)} \\
			&= 10 \text{ (mod 15)}
		\end{align*}
	\end{proof}
\end{solution}
\item $2x \equiv 5 \pmod{16}$
\begin{solution}
	No solution. 2 and 16 are not relatively prime, therefore $2x \equiv 5 \pmod{16} \implies 2x \equiv 1 \pmod{2}$, which has no solution, therefore the former equation is also inconsistent.

	
\end{solution}
\item $5x \equiv 10 \pmod{25}$
\begin{solution}
	2, 7, 12, 17, and 22 ( which are unique modulo 15 ).
	\begin{proof}
		All three numbers are divisible by 5, therefore the original equation implies \[x \equiv 2 \pmod{5}\]
		From there, we solve the equation using $(1^{-1} \pmod{5}=6)$:
		\begin{align*}
			1\cdot 1^{-1} \text{ (mod 5)} &= 2\cdot 1^{-1} \text{ (mod 5)} \\
			&= 2 \cdot 6 \text{ (mod 5)} \\
			&= 12 \text{ (mod 5)} \\
			&= 2 \text{ (mod 5)}
		\end{align*}
		Which is equivalent to \{2, 7, 12, 17, 22\} modulo 25.
	\end{proof}
\end{solution}
\end{enumerate}


\Question{Problem 2: Euclid's Algorithm}

\begin{enumerate}[label=(\alph*)]
    \item Use Euclid's algorithm from lecture to compute the greatest common divisor of 527 and 323. List the values of $x$ and $y$ of all recursive calls.
	\begin{solution} 17.
	\begin{center}
		\begin{tabular}{c|c}
		  $x$&$y$ \\
		  \hline
		  527&324 \\
		  323&204 \\
		  204&119 \\
		  119&85 \\
		  85&34 \\
		  34&17 \\
		  17&0
		\end{tabular}
	\end{center}

	\end{solution}
    \item Use extended Euclid's algorithm from lecture to compute the multiplicative inverse of 5 mod 27. List the values of $x$ and $y$ and the returned values of all recursive calls.
	\begin{solution} 11.
	\begin{center}
		\begin{tabular}{c|c}
		  $x$&$y$ \\
		  \hline
		  27&5 \\
		  5&2 \\
		  2&1 \\
		  1&0 \\
		\end{tabular}$\implies$
		\begin{tabular}{c|c|c}
		  $d$&$a$&$b$ \\
		  \hline
		  1&-2&11 \\
		  1&1&-2 \\
		  1&0&1 \\
		  1&1&0
		\end{tabular}
	\end{center}
	From the extended Euclid's algorithm, we know: $1=-2(27) + 11(5) \implies 5^{-1} \pmod{27}=11.$

	\end{solution}

    \item Find $x \pmod{27}$ if $5x + 2 6 \equiv 3 \pmod{27}$. You can use the result computed in (b).
	\begin{solution} 17 (mod 27).
	\begin{align*}
		5x + 2 6 &\equiv 3 	 \pmod{27} \\
		5x &\equiv -23 		 \pmod{27} \\
		5x &\equiv 4 		 \pmod{27} \\
		 x &\equiv 4\cdot 11 \pmod{27} \\
		 x &\equiv 17 		 \pmod{27}
	\end{align*}
	\end{solution}
    \item Assume $a$, $b$, and $c$ are integers and $c>0$. Prove or disprove: If $a$ has no multiplicative inverse mod $c$, then $ax \equiv b \pmod{c}$ has no solution.
   	\begin{solution}
   		False. Consider the equation $5x \equiv 10 \pmod{25}$. 5 has no multiplicative inverse mod 25. However, $x=2$ is a solution.
   	\end{solution}

\end{enumerate}


\Question{Problem 3: Modular Exponentiation}

Compute the following:

\begin{enumerate}[label=(\alph*)]

\item $13^{2018} \pmod {12}$
\begin{solution} 1.
	\begin{align*}
		13^1&=1\pmod{12}\\
		13^2&=1\pmod{12} \\
		&... \\
		13^{2018}&=\text{some product of ones (mod 12)}\\&=1\pmod{12}
	\end{align*}
\end{solution}
\item $8^{11111} \pmod 9$
\begin{solution} $8\pmod{9}$.
	\begin{align*}
		8^1&=8\pmod{9}\\
		8^2&=1\pmod{9} \\
		8^4&=1\pmod{9} \\
		&... \\
		8^{11111}&=8^1\cdot(8^2)^{5555}\pmod{9}\\
		&=8\cdot 1\pmod{9}\\
		&=8 \pmod{9}
	\end{align*}
\end{solution}
\item $7^{256} \pmod {11}$
\begin{solution} $4\pmod{11}$.
	\begin{align*}
		7^1&=7\pmod{11}\\
		7^2&=5\pmod{11} \\
		7^4&=3\pmod{11} \\
		7^8&=9\pmod{11} \\
		7^{16}&=4\pmod{11} \\
		7^{32}&=5\pmod{11} \\
		&... \\
		7^{256}&=(7^{32})^{2^3}\pmod{11}\\
		&=(7^{2})^{2^3}\pmod{11} \text{ (because it repeats)}\\
		&=(7^{16})\pmod{11}\\
		&=4 \pmod{11}
	\end{align*}
\end{solution}

\item $3^{160} \pmod {23}$
\begin{solution} 16 (mod 23).
	\begin{align*}
		3^1&=3 \pmod{23} \\
		3^2&=9 \pmod{23} \\
		3^4&=12 \pmod{23} \\
		3^8&=6 \pmod{23} \\
		3^{16}&=13 \pmod{23} \\
		3^{32}&=8 \pmod{23} \\
		3^{64}&=18 \pmod{23} \\
		3^{128}&=2 \pmod{23} \\
		&\\
		\therefore 3^{160}&=3^{128}\cdot3^{32}\pmod{23} \\
		&= 2\cdot 8 \pmod{23} \\
		&= 16 \pmod{23}
	\end{align*}

\end{solution}
\end{enumerate}

\Question{Problem 4: Euler's Totient Function}

  Euler's totient function is defined as follows:
  $$\phi(n) = | \{i: 1 \leq i \leq n, \texttt{gcd}(n,i) = 1\} |$$
  In other words, $\phi(n)$ is the total number of positive integers less than or equal to $n$ which are relatively prime to it.
  Here is a property of Euler's totient function that you can use without proof:

  For $m,n$ such that \texttt{gcd}($m,n$) = 1, $\phi(mn) = \phi(m) \cdot \phi(n)$.

  \begin{enumerate}[label=(\alph*)]
  \item
    Let $p$ be a prime number.
    What is $\phi(p)$?
    \begin{solution}
    	$p-1$
    \end{solution}

    \item
    Let $p$ be a prime number and $k$ be some positive integer.
    What is $\phi(p^k)$?
	\begin{solution}
		$p^k-p^{k-1}$ [Alternatively: $(p-1)p^{k-1}$]
	\end{solution}

    \item
    Let $p$ be a prime number and $a$ be a positive integer smaller than $p$.
    What is $a^{\phi(p)} \pmod{p}$?\\\emph{(Hint: use Fermat's Little Theorem.)}
    \begin{solution} 1 $\pmod{p}$. 
    	\begin{proof}
    		\begin{align*}
    			a^{\phi(p)} \pmod{p} &\equiv a^{p-1}\pmod{p} \\
    			&\equiv (a^p/a) \pmod{p} \\
    			&\text{Using Fermat's Little Theorem..} \\
    			&\equiv a / a \pmod{p} \\
    			&\equiv 1 \pmod{p}
    		\end{align*}
    	\end{proof}
    \end{solution}

	\newpage
    \item
    Let $b$ be a positive integer whose prime factors are $p_1,p_2,\hdots, p_k$.
    We can write $b = p_1^{\alpha_1}\cdot p_2^{\alpha_2} \hdots p_k^{\alpha_k}$.

    Show that for any $a$ relatively prime to $b$, the following holds:
    $$\forall i \in \{1,2,\hdots,k\}, \hspace{0.2cm} a^{\phi(b)} \equiv 1 \pmod{p_i}$$
    \begin{solution}
    \begin{proof}
    	
    	\begin{align}
    		b &= p_1^{\alpha_1}\cdot p_2^{\alpha_2} \hdots p_k^{\alpha_k} \\
    		\phi{(b)} &= \phi{(p_1^{\alpha_1})}\cdot \phi{(p_2^{\alpha_2})} \hdots \phi{(p_k^{\alpha_k})} \\
    		\phi{(b)} &= (p_1-1)(p_1^{\alpha_1 - 1}) \cdot (p_2 - 1)(p_2^{\alpha_2 - 1}) \hdots (p_k - 1)(p_k^{\alpha_k - 1}) \label{eq:1}
    	\end{align}
    	From here, we want to convert $\phi{(b)}$ into a form we can use with our extension to Fermat's Little Theorem from part (c). More specifically we want to isolate the $(p_i -  1)$ term.
    	
    	Conveniently, all of the terms in \eqref{eq:1} are natural numbers, therefore we can rewrite all other terms of \eqref{eq:1} as a product of constants:
    	\begin{align}
    		\phi{(b)} &= (p_i - 1)\cdot \prod_{j}{c_j} \\
    				  &= (p_i - 1)\cdot c \label{eq:2}
    	\end{align}

		From \eqref{eq:2}, we can solve for $a^{\phi{(b)}} \pmod{p_i}$:
		\begin{align}
			a^{\phi(b)} &\equiv a^{(p_i -1)\cdot c} \pmod{p_i} \\
			&\equiv  (a^{p_i -1} \pmod{p_i})^{c} \\
			&\equiv 1^c \pmod{p_i} \\
			&\equiv 1 \pmod{p_i}
		\end{align}
		
		This works for any arbitrary $p_i$, therefore the initial statement holds.
	    \end{proof}

    \end{solution}

  \end{enumerate}


\Question{Problem 5: FLT Converse}

Recall that the FLT states that, given a prime $n$, $a^{n-1} \equiv 1 \pmod{n}$ \textit{for all $1 \leq a \leq n-1$}. Note that it says nothing about when $n$ is composite.

Can the FLT condition ($a^{n-1} \equiv 1 \mod n$) hold for some or even all $a$ if $n$ is composite? This problem will investigate both possibilities. 
It turns out that unlike in the prime case, we need to restrict ourselves to looking at $a$ that are relatively prime to $n$. (Note that if $n$ is prime, then every $a<n$ is relatively prime to $n$). Because of this restriction, let's define $$S(n) =  \{i: 1 \leq i \leq n, \texttt{gcd}(n,i) = 1\},$$ so $\mid S \mid$ is the total number of possible choices for $a$.
\begin{enumerate}[label=(\alph*)]
\item Prove that for every $a$ and $n$ that are not relatively prime, FLT condition fails. In other words, for every $a$ and $n$ such that $\texttt{gcd}(n,a) \neq 1$, we have $a^{n-1} \not\equiv 1 \pmod{n}$.
	\begin{solution}
		\begin{proof}
			$a^{n-1} \equiv 1 \pmod{n} 	\iff \exists x:a^{n-1}x\equiv 1\pmod{n} \iff a^{n-1}$ has an inverse $\pmod{n}$. Since we know $n$ and $a$ are not relatively prime, we also know $\texttt{gcd}(n,a^{n-1}) \neq 1$ (the prime factorization of $a^{n-1}$ would contain the same common prime factor of $n$ and $a$). From there, we know from Theorem 6.2 that $a^{n-1}$ has no inverse, therefore $a^{n-1} \not\equiv 1 \pmod{n}$
		\end{proof}
	\end{solution}

    \item Prove that the FLT condition fails for most choices of $a$ and $n$. More precisely, show that if we can find a single $a \in S(n)$ such that $a^{n-1} \not \equiv 1 \pmod{n}$, we can find at least $|S(n)|/2$ such $a$. (Hint: You're almost there if you can show that the set of numbers that fail the FLT condition is at least as large as the set of numbers that pass it. A clever bijection may be useful to compare set sizes.)
	\begin{solution}
		
	\end{solution}
\end{enumerate}
The above tells us that if a composite number fails the FLT condition for even one number relatively prime to it, then it fails the condition for most numbers relatively prime to it. However, it doesn't rule out the possibility that some composite number $n$ satisifes the FLT condition entirely: \emph{for all} $a$ relatively prime to $n$, $a^{n-1}\equiv 1 \mod n$. It turns out such numbers do exist, but they were found through trial-and-error! We will prove one of the conditions on $n$ that make it easy to verify the existence of these numbers.
\begin{enumerate}[label=\alph*.)]\setcounter{enumi}{2}
    \item First, show that if $a \equiv b \mod m_1$ and $a \equiv b \mod m_2$, with $\gcd(m_1, m_2)=1$, then $a \equiv b \pmod{m_1 m_2}$.
	\begin{solution}
	\begin{proof}
	Starting from the givens:
		\begin{align*}
			a \equiv b \mod m_1 &\implies m_1 | (b-a)\\
			a \equiv b \mod m_2 &\implies m_2 | (b-a)\\
		\end{align*}
	From here, we can proceed by contradiction. 
	
	If $a \not\equiv b \pmod{m_1 m_2}$, then $m_1 m_2\not|(b-a)$
	
	$\implies b-a=m_1m_2k+r$ (where $k\in \Z$ and $r \in \Z^+, r<m_1m_2$)
	
	$\implies r = (b-a) -m_1m_2k$
	
	Now we know $m_1 | (b-a)$ and $m_2 | (b-a)$  
	$\implies r = m_1c_1 -m_1m_2k$ and $r = m_2c_2 -m_1m_2k$ for some $c_1,c_2$. This implies $m_1|r$ and $m_2|r$. Since $\gcd(m_1, m_2)=1$, the lowest number that is divisible by both is $m_1m_2$. However, we know $r<m_1m_2$, therefore we have a contradiction as desired.
	\end{proof}
	\end{solution}

    \item Let $n = p_1 p_2 \cdots p_k$ where $p_i$ are distinct primes and $p_i - 1 \mid n - 1$ for all $i$. Show that $a^{n-1} \equiv 1 \pmod{n}$ for all $a \in S(n)$ 
	\begin{solution}
		\begin{proof}
			Starting from the given: 
			
			$a\in S(n) \implies gcd(a,n)=1$.
			
			For an arbitrary $p_i, a^{n-1}\equiv 1 \pmod{p_i}$ (from: $p_i - 1 \mid n - 1$).
			
			From (c), this implies $a^{n-1}\equiv 1 \pmod{p_1p_2...p_k}\equiv 1 \pmod{n}$
		\end{proof}
	\end{solution}

    
    %relatively prime to $n$.
    
        % \item Now we will show the converse; if a composite number $n$ satisfies the FLT condition, it must be the product of primes $p_i$ where $p_i \mid n-1$

    %     \item First, show that for every prime $p$, you can find a number $a$ such that $a^{p-1} = 1 \mod p$ \emph{and} $a^k \neq 1 \mod p$ for all $1 \leq k < p-1$. We call $p-1$ the \textit{order} of $a \mod p$

    \item Verify that for all $a$ coprime with 561, $a^{560} \equiv 1 \pmod{561}$.
	\begin{solution} If we can verify $p_i - 1 \mid 561 - 1$ for all prime factors of 561, then from (d) we know $a^{560} \equiv 1 \pmod{561}$ for all $a$ coprime with 561.
	
		Prime factors of 561 = \{3, 11, 17\}

		$(2|560)\land (10|560) \land (16|560) \implies$$a^{560} \equiv 1 \pmod{561}$ for all $a$ coprime with 561.
		
	\end{solution}

\end{enumerate}




\end{document}
