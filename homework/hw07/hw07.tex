\documentclass[11pt, notitlepage]{report}

	\usepackage[margin=1in]{geometry}
	\usepackage{amsmath,amsthm,amssymb,amsfonts}
	\usepackage{enumitem}
	\usepackage{systeme}

	\newcommand{\N}{\mathbb{N}}
	\newcommand{\Z}{\mathbb{Z}}
	\newcommand{\R}{\mathbb{R}}
	\newcommand{\A}{\alpha}
	\newcommand{\ora}[1]{\overrightarrow{#1}}
	\newcommand{\Question}[1]{\newpage\section{#1}}
	\usepackage[parfill]{parskip}
	\usepackage{mathtools}
	\newenvironment{solution}{\paragraph{Solution:}}{\hfill}
	\newenvironment{theorem}{\paragraph{Theorem:}}{\hfill}
	\newenvironment{subtheorem}[1]{\paragraph{\small Subtheorem #1:}}{\hfill}
	\newenvironment{definition}{\paragraph{Definition:}}{\hfill}
	\newenvironment{problem}[2][Problem]{\begin{trivlist}
	\item[\hskip \labelsep {\bfseries #1}\hskip \labelsep {\bfseries #2.}]}{\end{trivlist}}
	
	\usepackage{pgfplots}
	\usetikzlibrary{arrows}
	\usetikzlibrary{decorations.markings}
	\usetikzlibrary{datavisualization}
	\usetikzlibrary{datavisualization.formats.functions}
	%\usepackage{pstricks-add}
	\setlength\parindent{24pt}
	\pgfplotsset{every axis/.append style={
	                   axis x line=middle,    % put the x axis in the middle
	                   axis y line=middle,    % put the y axis in the middle
	                   axis line style={<->,color=gray}, % arrows on the axis
	                   xlabel={$x$},          % default put x on x-axis
	                   ylabel={$y$},          % default put y on y-axis
	           }}
	\pgfplotsset{compat=1.15}
	
	\newcommand{\pgraph}[4]{
		\begin{center}
		
		\begin{tikzpicture}
		\begin{axis}[
		   trig format plots=rad,
		   axis equal,
		   grid=both
		]
		\addplot [domain=#3:#4, variable=\t, samples=50, black, decoration={
		   markings,
		   mark=between positions 0.2 and 1 step 4em with {\arrow [scale=1.5]{stealth}}
		   }, postaction=decorate]
		({#1}, {#2});
		
		\end{axis}
		\end{tikzpicture}
		
		\end{center}
	}


   \newcommand{\cgraph}[3]{
	   \begin{center}
	
	   \begin{tikzpicture}
	   \begin{axis}[
	       trig format plots=rad,
	       axis equal,
	       grid=both
	   ]
	   \addplot [domain=#2:#3, variable=\x, samples=50, black, decoration={
	       markings,
	       mark=between positions 0.2 and 1 step 4em with {\arrow [scale=1.5]{stealth}}
	       }, postaction=decorate]
	   {#1};
	
	   \end{axis}
	   \end{tikzpicture}
	
	   \end{center}
	}


	
	\makeatletter
	\newcommand*{\toccontents}{\@starttoc{toc}}
	\makeatother


\begin{document}
   \title{CS 70: Homework \#6}
   \author{Abhijay Bhatnagar}
   \maketitle
   \toccontents



\setcounter{secnumdepth}{0} %% no numbering

\Question{Bijective or not?}

Decide whether the following functions are bijections or not. Please prove your claims.
\begin{enumerate}[label=\alph*.)]

    \item $f(x) = 10^{-5}x$
    \begin{enumerate}[label=(\roman*)]
        \item for domain $\mathbb R$ and range $\mathbb R$
        \begin{solution}
        	Yes.
        	\begin{proof} Let $f$ : $\R\rightarrow\R$.
        		
        		One-to-one: Assume there exists $f(x)=f(y)$. That implies $10^{-5}x=10^{-5}y\implies x=y$, therefore $f(x)=f(y)\implies x=y$ and $f$ is one-to-one.
        		
        		Onto: If $y\in \R$, then $f(10^5y)=y$, therefore $f$ is onto.
        		
        	\end{proof}
        \end{solution}
        \item for domain $\mathbb Z \cup \{\pi\}$ and range $\mathbb R$
        \begin{solution}
        	No.
        	\begin{proof} Let $f$ : $\mathbb Z \cup \{\pi\}\rightarrow\R$.
        		
        		To disprove the bijective, all I need is one counter example.
        		
        		Consider $y=\frac{2.5}{10^{5}}$. There exists no $x\in \mathbb Z \cup \{\pi\}$ s.t. $f(x)=\frac{2.5}{10^{5}}$ (because that would imply $x=2.5\implies x\not\in \mathbb Z \cup \{\pi\}$). This means $f$ is not onto, and therefore is not bijective.
        	\end{proof}
        \end{solution}
    \end{enumerate}

    \item $f(x) = p \bmod x$, where $p > 2$ is a prime
    \begin{enumerate}[label=(\roman*)]
        \item for domain $\mathbb N \setminus \{0\}$ and range $\{0, \dots, p\}$ 
        \begin{solution} No.
       	\begin{proof}
       		Let $f : \N \setminus \{0\} \rightarrow \{0,...,p\}$. 
       		
       		$(\forall x: x\in \N \setminus \{0\}), (x>p) \implies f(x)=p \implies$ $f$ is not injective, therefore cannot be bijective. (Additionally $\not \exists x : p \bmod x = p$, therefore it is also not surjective.)
       	\end{proof}
        \end{solution}
        \item for domain $\{(p+1)/2, \dots, p\}$ and range $\{0, \dots,
            (p-1)/2\}$
        \begin{solution} Yes.
        	\begin{proof}
        		Let $f : \{(p+1)/2, (p+3)/2, \dots, p\} \rightarrow \{0, \dots, (p-3)/2, (p-1)/2\}$ (slightly more clear articulation of given domain and range).
        		
        		We can prove the injective and surjective properties at the same time. 
        		
        		Each element $x$ in the given domain is greater than half of $p$. This implies that $f(x) = p \bmod x = p-x$, which is unique for each distinct x, implying the injective property holds. Extending this to the elements within the domain, we can see the range is: \begin{align*}
        		 &\{p-\frac{p+1}{2}, p-\frac{p+3}{2}, \dots, p-\frac{p+p}{2}\} \\
        		 &=\{\frac{2p - (p+1)}{2}, \frac{2p -(p+3)}{2}, \dots, \frac{2p -(p+p)}{2}\} \\
        		 &=\{\frac{p-1}{2}, \frac{p-3}{2}, \dots, 0\}
        		\end{align*} 
        		which is equivalent to our range (except in reverse order). Therefore $f$ is also surjective, implying $f$ is bijective.
        	\end{proof}
        \end{solution}
    \end{enumerate}

    \item $f(x) = \{x\}$, where the domain is $D = \{0,\dots, n\}$ and the range 
    is $\mathcal P(D)$, the powerset of $D$ (that is, the set of all subsets of
    $D$).
    \begin{solution}
    	No. 
    	\begin{proof}
    		Consider any subject of $D$ with a length greater than one, and let us call it $y$. $f(x)$ will only produce subsets of length 1, therefore it will never hit $y$, therefore the function is not surjective (and not bijective).
    	\end{proof}
    \end{solution}
    \item Consider the number $X = 1234567890$, and obtain $X'$ by shuffling the
    order of the digits of $X$. Is $f(i) = (\text{i} + 1)^{\text{st}}$ \textit{digit
    of} $X'$ a bijection from $\{0,\dots, 9\}$ to itself?
    \begin{solution}Yes.
    	\begin{proof}
    		
    		Injective: $X'$ has 10 digits, each of which are distinct numbers from $0-9$. Therefore each $(\text{i} + 1)^{\text{st}}$ digit will output a distinct number.
    		
    		Surjective: $\{0,\dots, 9\}$ are all contained within the digits of $X'$. We've shown, each of the ten outputs is injective, therefore there must be ten distinct outputs, each of must be an element of $\{0,\dots, 9\}$, which has ten elements, implying all of the elements of the set are reached (potentially in a different order). Therefore the function is also surjective.
    	\end{proof}
    \end{solution}
\end{enumerate}


\Question{Counting Tools}
Are the following sets countable or uncountable? Please prove your claims.
\begin{enumerate}[label=\alph*.)]
    \item $A \times B$, where $A$ and $B$ are both countable.
    \begin{solution} Countable. Construct a grid of points $(A_i, B_j)$, we can construct a mapping from $\N$ such that the entire grid is covered through diagonal stripes, i.e. we go from $(A_0, B_0)\rightarrow (A_0, B_1)\rightarrow (A_1, B_0) \rightarrow (A_2, B_0) \rightarrow (A_2, B_1) \rightarrow (A_2, B_2)$, and map the natural numbers to each step in that path. This is injective and surjective with $\N$ since each point is unique and it covers all points, therefore it is countable.
    \end{solution}
    \item $\bigcup_{i\in A} B_i$, where $A, B_i$ are all countable.
   	\begin{solution}
   		Countable. We can construct a grid where the rows are the different $B_i$ and the columns are the elements of each $B_i$, i.e. $B_{ij}$. Since $A, B_i$ are all countable, $B_i$ is injective with $\N$, and $\forall i, B_{ij}$ is injective with $\N$ (with respect to j), we can construct a similar diagonal path on the grid as part (a) that allows us to map the natural numbers to the different points on the diagonal paths.
   	\end{solution}
    \item The set of all functions $f$ from $\N$ to $\N$ such that $f$ is
    non-decreasing. That is, $f(x) \leq f(y)$ whenever $x \leq y$.
    \begin{solution}
    	Uncountable. 
    	\begin{proof}
    		We can proceed with diagonalization. Let $A$ = the set given in the problem. 
    		
    		Let us assume $A$ is countable. Since $A$ is countable, we can enumerate each $f\in A$ with the natural numbers, i.e. $f_1, f_2, \dots$. Let us construct a new function $f'(n)= {f_n(n+1)-f_n(n) + f(n-1) + 1}, f'(1)=1$, which is an increasing function, modified slightly. If $f'$ existed in $A$, then we code denote it $f_i$, but because of how we used $f_i$ and a modification in the construction of $f'$, we know $f_i(i)$ and $f'(i)$ must differ, therefore $f'\not\in A \implies A$ is not countable.
    	\end{proof}
    \end{solution}
    \item The set of all functions $f$ from $\N$ to $\N$ such that $f$ is
    non-increasing. That is, $f(x) \geq f(y)$ whenever $x \leq y$.
	\begin{solution}
		Countable.
		\begin{proof}
			We can construct a combination of countable many countable sets.
		\end{proof}
	\end{solution}    
	
	\newpage
	
    \item The set of all bijective functions from $\N$ to $\N$.
    \begin{solution} Uncountable.
    \begin{proof}
    	Let A = set of all bijective functions from $\N \rightarrow \N$. We can enumerate each of these functions in $A$ as $f_i$. Then we can construct a new function such that $f(n)$ is some $x\in N$ such that $f_n(n) \not = x,$ and $\forall m < n, f(m) \not = x. $ This is clearly not within $A$ (since the terms are defined exactly such that $f_n(n)$ differs from $f(n)$, therefore A is uncountable.
    \end{proof}
    \end{solution}
\end{enumerate}



\Question{Impossible Programs}

Show whether the following programs can exist or not.

\begin{enumerate}[label=\alph*.)]

\item A program $P$ that takes in any program $F$, input $x$ and output $y$ and returns 
true if $F(x)$ outputs $y$ and false otherwise.
\begin{solution}Impossible.
\begin{proof} Let \texttt{Program} be the program defined by the question.

	We can reduce \texttt{TestHalt} to \texttt{Program}
	
	\texttt{TestHalt(P, x, y): \\
	\phantom{---}P'(x): \\
	\phantom{---}\phantom{---} run P(x) while suppressing outputs\\
	\phantom{---}\phantom{---} return y\\
	\phantom{---}if Program(P',x,y): \\
	\phantom{---}\phantom{---} return True \\
	\phantom{---}else: \\
	\phantom{---}\phantom{---} return False \\
	}
	
\end{proof}
\end{solution}
%\item A program $P$ that takes in any program $F$, input $x$ and line number
%$L$, and returns true if line $L$ is ever executed when $F$ is run on $x$, and
%false otherwise.

\item A program $P$ that takes in two programs $F$ and $G$ and returns true if $F$ and $G$ 
halt on the same inputs, and false otherwise.
Impossible.
\begin{solution}\begin{proof} Let \texttt{Program} be the program defined by the question.

	We can reduce \texttt{TestEasyHalt} to \texttt{Program}
	
	\texttt{TestEasyHalt(F): \\
	\phantom{---}P(x): \\
	\phantom{---}\phantom{---} if x = 0: return halt\\
	\phantom{---}\phantom{---} else: return F(x)\\
	\phantom{---}if Program(F,P): \\
	\phantom{---}\phantom{---} return True \\
	\phantom{---}else: \\
	\phantom{---}\phantom{---} return False \\
	}
	
\end{proof}
\end{solution}
\end{enumerate}


\Question{Undecided?}

Let us think of a computer as a machine which can be in any of $n$ states
$\{s_1, \dots, s_n\}$. The state of a $10$ bit computer might for instance be
specified by a bit string of length $10$, making for a total of $2^{10}$ states
that this computer could be in at any given point in time. An algorithm
$\mathcal A$ then is a list of $k$ instructions $(i_0, i_2, \dots, i_{k-1})$, where each 
$i_l$ is a function of a state $c$ that returns another state $u$ and a number $j$.
Executing $\mathcal A(x)$ means computing 

\[ (c_1,j_1) = i_0(x), \hspace{10mm} (c_2,j_2) = i_{j_1}(c_1), \hspace{10mm} (c_3, j_3) =
i_{j_2}(c_2), \hspace{10mm} \dots  \]

until $j_{\ell} \geq k$ for some $\ell$, at which point the algorithm halts and returns
$c_{\ell-1}$.

\begin{enumerate}[label=\alph*.)]
    \item How many iterations can an algorithm of $k$ instructions perform on an $n$-state 
    machine (at most) without repeating any computation?
    \begin{solution}
    	$n\cdot k$ iterations. There are $k$ distinct $j$ numbers and $n$    \end{solution}
    \item Show that if the algorithm is still running after $2n^2k^2$
    iterations, it will loop forever.
    \begin{solution}
    	We can prove a stronger case, i.e. if it is still running at $nk$ ($nk < 2n^2k^2$), it will loop forever. Mainly, we know that it can perform at max $nk$ computations before it starts repeating, therefore if it is still running after that many iterations it must be repeating. If it is repeating, it will restart the sequence at the corresponding point within the first $nk$ iterations.
    \end{solution}
    \item Give an algorithm that decides whether an algorithm $\mathcal A$ halts
    on input $x$ or not. Does your construction contradict the undecidability of
    the halting problem?
    \begin{solution} \
    
    	\texttt{
    	TestHalt(A, x, k, n): \\
    	\phantom{---} let iters = 1 \\
    	\phantom{---} c, j = i(x) \\
    	\phantom{---} while j < k: \\
    	\phantom{---}\phantom{---} c,j = run next step of A(x) \\
    	\phantom{---}\phantom{---} iters+=1 \\
    	\phantom{---}\phantom{---} if iters > n*k: \\
    	\phantom{---}\phantom{---} \phantom{---} return False \\
    	\phantom{---} return True
    	}
    \end{solution}
\end{enumerate}


\Question{Clothing Argument}

  \begin{enumerate}[label=\alph*.)]

      \item There are four categories of clothings (shoes, trousers, shirts,
      hats) and we have ten distinct items in each category. How many distinct
      outfits are there if we wear one item of each category?
      \begin{solution}
      	$10^4$. There are 10 options for 4 categories, therefore $10\times 10 \times 10 \times 10$ different combinations.
      \end{solution}
      \item How many outfits are there if we wanted to wear exactly two categories?
	  \begin{solution}
	  	$6 \times 10^2$. 
	  	
	  	For two categories, there are $10^2$ different combinations of items, and we have to choose 2 categories out of 4. This gives us a total =
	  	(4 choose 2)(10 x 10) = $\frac{4!}{2!2!}\times 10^2$. (Intuition for 4 choose 2 is that for each of the four categories you have 3 choices, giving $4\time 3$, however each pairing gets duplicated when you start with the other pair, therefore divide by $2$.)
	  \end{solution}
      \item How many ways do we have of hanging four of our ten hats in a row on
      the wall? (Order matters.)
	  \begin{solution}
	  	5040. This is a simple permutation problem. $\frac{10!}{(10-4)!}=7\times8\times9\times10$
	  \end{solution}
      \item We can pack four hats for travels. How many different possibilities for packing 
      four hats are there? Can you express this number in terms of your answer
      to part (c)?
      \begin{solution}
      	1260 (equivalent to 5040/4!). This is simple combination problem, analogous to taking (c) and eliminating the duplicates.
      \end{solution}
      \item If we have exactly $3$ red hats, $3$ green hats and $4$ turquoise
      hats, and hats of the same colour are indistinguishable, how many distinct
      sets of three hats can we pack?
	  \begin{solution}
	  	10. ${3+3-1\choose 3} = {5\choose 3} = \frac{5!}{3!2!}$
	  \end{solution}
  \end{enumerate}


\end{document}