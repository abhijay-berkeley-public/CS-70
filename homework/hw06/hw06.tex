\documentclass[11pt, notitlepage]{report}

	\usepackage[margin=1in]{geometry}
	\usepackage{amsmath,amsthm,amssymb,amsfonts}
	\usepackage{enumitem}
	\usepackage{systeme}

	\newcommand{\N}{\mathbb{N}}
	\newcommand{\Z}{\mathbb{Z}}
	\newcommand{\R}{\mathbb{R}}
	\newcommand{\A}{\alpha}
	\newcommand{\ora}[1]{\overrightarrow{#1}}
	\newcommand{\Question}[1]{\newpage\section{#1}}
	\usepackage[parfill]{parskip}
	\usepackage{mathtools}
	\newenvironment{solution}{\paragraph{Solution:}}{\hfill}
	\newenvironment{theorem}{\paragraph{Theorem:}}{\hfill}
	\newenvironment{subtheorem}[1]{\paragraph{\small Subtheorem #1:}}{\hfill}
	\newenvironment{definition}{\paragraph{Definition:}}{\hfill}
	\newenvironment{problem}[2][Problem]{\begin{trivlist}
	\item[\hskip \labelsep {\bfseries #1}\hskip \labelsep {\bfseries #2.}]}{\end{trivlist}}
	
	\usepackage{pgfplots}
	\usetikzlibrary{arrows}
	\usetikzlibrary{decorations.markings}
	\usetikzlibrary{datavisualization}
	\usetikzlibrary{datavisualization.formats.functions}
	%\usepackage{pstricks-add}

	\pgfplotsset{every axis/.append style={
	                   axis x line=middle,    % put the x axis in the middle
	                   axis y line=middle,    % put the y axis in the middle
	                   axis line style={<->,color=gray}, % arrows on the axis
	                   xlabel={$x$},          % default put x on x-axis
	                   ylabel={$y$},          % default put y on y-axis
	           }}
	\pgfplotsset{compat=1.15}
	
	\newcommand{\pgraph}[4]{
		\begin{center}
		
		\begin{tikzpicture}
		\begin{axis}[
		   trig format plots=rad,
		   axis equal,
		   grid=both
		]
		\addplot [domain=#3:#4, variable=\t, samples=50, black, decoration={
		   markings,
		   mark=between positions 0.2 and 1 step 4em with {\arrow [scale=1.5]{stealth}}
		   }, postaction=decorate]
		({#1}, {#2});
		
		\end{axis}
		\end{tikzpicture}
		
		\end{center}
	}


   \newcommand{\cgraph}[3]{
	   \begin{center}
	
	   \begin{tikzpicture}
	   \begin{axis}[
	       trig format plots=rad,
	       axis equal,
	       grid=both
	   ]
	   \addplot [domain=#2:#3, variable=\x, samples=50, black, decoration={
	       markings,
	       mark=between positions 0.2 and 1 step 4em with {\arrow [scale=1.5]{stealth}}
	       }, postaction=decorate]
	   {#1};
	
	   \end{axis}
	   \end{tikzpicture}
	
	   \end{center}
	}


	
	\makeatletter
	\newcommand*{\toccontents}{\@starttoc{toc}}
	\makeatother


\begin{document}
   \title{CS 70: Homework \#6}
   \author{Abhijay Bhatnagar}
   \maketitle
   \toccontents



\setcounter{secnumdepth}{0} %% no numbering

\Question{Problem 1: Polynomial Practice}
\begin{enumerate}[label=\alph*)]
	\item If $f$ and $g$ are non-zero real polynomials, how many roots do the following 
    polynomials have at least? How many can they have at most? 
    (Your answer may depend on the degrees of $f$ and $g$.)
	\begin{enumerate}[label=\roman*.)]
        \item $f + g$ 
        \begin{solution}
        	max(degree of $f$, degree of $g$). 
        	
        	Addition cannot change polynomial exponents.
        \end{solution}
        \item $f\cdot g$
        \begin{solution}
        	degree of $f$ + degree of $g$. 
        	
        	The highest ordered term in multiplication is a combination of the highest ordered terms of the polynomials.
        \end{solution}
        \item $f/g$, assuming that $f/g$ is a polynomial.        \begin{solution}
        	degree of $f$ - degree of $g$.  
        	
        	The highest ordered term in division is a subtraction of the highest ordered terms of the polynomials.

        \end{solution}
    \end{enumerate}

	\item Now let $f$ and $g$ be polynomials over $\mathrm{GF}(p)$.
    \begin{enumerate}[label=\roman*.)]
        \item If $f\cdot g = 0$, is it true that either $f=0$ or $g=0$?
        \begin{solution}
%        	Yes. If $f(x)\cdot g(x) = 0$ for all $x$, it means for every $x$ each term of each power cancels out evenly. 
		Not necessarily. Consider $f=x+1$ and $g=x+2$ in $GF(2)$. $f \cdot g = 0$, but $f(0)=1$ and $g(1)=1$. 
        \end{solution}
        \item If $\deg{f} \geq p$, show that there exists a polynomial $h$ with 
            $\deg{h} < p$ such that $f(x) = h(x)$ for all $x \in \{0,1,...,p-1\}$.
        \begin{solution}\begin{proof}
        	Within $GF(p)$, all solutions of $f(x)$ are cyclical, i.e. $S_f = \{f(1), f(2),...,f(n):0\leq n<p\}$, giving us $p$ solutions. From those, we can construct $p$ pairs of points $(i, f(i))$, and use Lagrangian interpolation to find a function of degree $p-1$ that exactly fits those points.
        \end{proof}\end{solution}

	\newpage
        \item How many $f$ of degree \textit{exactly} $d<p$ are there such that 
            $f(0) = a$ for some fixed $a\in\{0,1,\dots,p-1\}$?
            
            \begin{solution}
            	\begin{proof}
            		$m^d$ polynomials. We are given only 1 point, $(0, a)$, and so we have $d$ points of freedom.
            	\end{proof}
            \end{solution}
    \end{enumerate}

    \item Find a polynomial $f$ over $\mathrm{GF}(5)$ that satisfies 
    $f(0) = 1, f(2) = 2, f(4) = 0$. How many such polynomials are there?
    \begin{solution}
    	$f=\frac{-3}{8}x+\frac{5}{4}x+1$. We are given three points, and you need five points to describe a degree 4 polynomial, so we have 2 points of freedom, meaning there are $5^2=25$ different polynomials.
    \end{solution}
\end{enumerate}
    




\Question{Problem 2: The CRT and Lagrange Interpolation}

Let $n_1, \ldots n_k$ be pairwise coprime, i.e. $n_i$ and $n_j$ are coprime for all $i \neq j$. The Chinese Remainder Theorem (CRT) tells us that there exist solutions to the following system of congruences:
\begin{align}
    x &\equiv a_1 \pmod{n_1} \tag{1} \\
    x &\equiv a_2 \pmod{n_2} \tag{2} \\
    &\vdots \tag{$\vdots$} \\
    x &\equiv a_k \pmod{n_k} \tag{$k$}
\end{align}
and all solutions are equivalent $\pmod{n_1 n_2 \cdots n_k}$. For this problem, parts (a)-(c) will walk us through a proof of the Chinese Remainder Theorem. 
We will then use the CRT to revisit Lagrange interpolation.

\begin{enumerate}[label=\alph*)]
\item We start by proving the $k=2$ case: Prove that we can always find an integer $x_1$ 
that solves $(1)$ and $(2)$ with $a_1 = 1, a_2 = 0$. Similarly, prove that we can always 
find an integer $x_2$ that solves $(1)$ and $(2)$ with $a_1 = 0, a_2 = 1$.
\begin{solution}
	\begin{proof}
		Let $n=2$. 
		\begin{enumerate}[label=(\roman*)]
		\item $a_1 = 1, a_2 = 0$
		\begin{align}
			x_1&\equiv 1 \pmod{n_1} \\
			x_1&\equiv 0 \pmod{n_2} \\
		\end{align}
		
		$x_1 = n_2 * (n_1 + 1)$ will always satisfy those constraints. ($x_1 = n_2\pmod{n_1n_2}$)
		\item $a_1 = 0, a_2 = 1$
		\begin{align}
			x_2&\equiv 0 \pmod{n_1} \\
			x_2&\equiv 1 \pmod{n_2} \\
		\end{align}
		
		$x_2 = n_1 * (n_2 + 1)$ will always satisfy those constraints. It is also unique ($x_2 = n_1\pmod{n_1n_2}$)

		\end{enumerate}
	\end{proof}
\end{solution}
\newpage
\item Use part (a) to prove that we can always find at least one solution to $(1)$ and $(2)$ for any $a_1,a_2$. Furthermore, prove that all possible solutions are equivalent 
$\pmod{n_1n_2}$. 
\begin{solution}
	\begin{proof}
		Any solution of the form $x=(n_1+a_1)(n_2+a_2) \pmod{n_1}$ will work.
		
	\end{proof}
\end{solution}

\item Now we can tackle the case of arbitrary $k$: Use part (b) to prove that there exists
a solution $x$ to $(1)$-$(k)$ and that this solution is unique $\pmod{n_1 n_2 \cdots n_k}$.
\begin{solution}
	Using the proof from (b) [if I had one], 
\end{solution}
\item For two polynomials $p(x)$ and $q(x)$, mimic the definition of $a \bmod{b}$ for 
integers to define $p(x) \bmod{q(x)}$. Use your definition to find $p(x) \bmod{(x-1)}$.

\item Define the polynomials $x-a$ and $x-b$ to be coprime if they have no common divisor 
of degree $1$. Assuming that the CRT still holds when replacing $x, a_i$ and $n_i$ with 
polynomials (using the definition of coprime polynomials just given), show that the system 
of congruences
\begin{align}
    p(x) & \equiv y_1 \pmod{(x-x_1)} \tag{1'} \\
    p(x) & \equiv y_2 \pmod{(x-x_2)} \tag{2'} \\
    &\vdots \tag{$\vdots$} \\
    p(x) & \equiv y_k \pmod{(x-x_k)} \tag{k'}
\end{align}
has a unique solution $\pmod{(x-x_1)\cdots(x-x_k)}$ whenever the $x_i$ are pairwise 
distinct. What is the connection to Lagrange interpolation?

\end{enumerate}


\Question{Problem 3: Old secrets, new secrets}

In order to share a secret number $s$, Alice distributed the values
$\left(1,p(1)\right), \left(2,p(2)\right), \dots, \left(n+1,p(n+1)\right)$ of a degree $n$ 
polynomial $p$ with her friends $\text{Bob}_1, \dots, \text{Bob}_{n+1}$. As usual, 
she chose $p$ such that $p(0) = s$. $\text{Bob}_1$ through $\text{Bob}_{n+1}$ now gather to 
jointly discover the secret. Suppose that for some reason $\text{Bob}_1$ already knows $s$,
and wants to play a joke on $\text{Bob}_2,\dots, \text{Bob}_{n+1}$, making them
believe that the secret is in fact some fixed $s' \neq s$. How can he achieve this?
\begin{solution}
	Bob uniquely knows $p(1)$, while all others know the $x_1$ and $\Delta_{1}(x)$ which implies $y_1=p(1)$ is the only thing he can manipulate. The final solution is the polynomial constructed by $\sum_{i}{y_i \Delta_i}$, implying he can manipulate the $y_1 \Delta_{1}(x)$ term. 
	
	We can look at the final solution in term of the deltas, i.e.:
	\begin{align*}P(x)&=a_{n}x^{n} + a_{n-1}x^{n-1} +...+a_1x+s\\
	&= p(1)\Delta_1(x) + p(2)\Delta_2(x) + ... +p(n+1)\Delta_{n+1}(x)\\
	&= p(1)\Delta_1(x) + \sum_{i\not=1}p(i)\Delta_i(x)
	\end{align*}
	
	From here we see how the polynomial is a sum of its Lagrangian parts:
	\[ P(x)=a_{n}x^{n} + a_{n-1}x^{n-1} +...+a_1x+s = p(1)\Delta_1(x) + \sum_{i\not=1}p(i)\Delta_i(x)\]
	However, since we are only concerned with altering the $P(0)$ term, all we need to do is consider the constants of the sub-polynomials, i.e.: 
		\[P(0) = s = p(1)\Delta_1^c(x) + \sum_{i\not=1}p(i)\Delta_i^c(x)\]
	We can determine $\Delta_1^c(x)=\frac{(x-2)(x-3)\cdot...\cdot(x-(n+1))}{(1-2)(1-3)\cdot...\cdot(1-(n+1))} = \frac{(n+1)!}{n!}=n+1$.
	
	From here we need to construct a $p'(1)$ value s.t. $p'(1)\Delta_1^c(x)=p(1)\Delta_1^c(x)+s-s'$ (add $s'$ and subtract $s$ from both sides to make the algebra work out).
	
	$\implies p'(1)=\frac{p(1)\Delta_1^c(x)+s-s'}{\Delta_1^c(x)}=p(1)+\frac{s-s'}{n+1}$.
	
	We simply provide that $P'(1)$ value to deceive the others.
\end{solution}

\Question{Problem 4: Berlekamp-Welch for General Errors}

Suppose that Hector wants to send you a length $n=3$ message, $m_0,m_1,m_2$, with the possibility for $k=1$ error. For all parts of this problem, we will work mod 11, so we can encode 11 letters as shown below:
$$\begin{tabular}{|ccccccccccc|}
\hline
A & B & C & D & E & F & G & H & I & J & K \\
\hline
0 & 1 & 2 & 3 & 4 & 5 & 6 & 7& 8 & 9 & 10 \\
\hline
\end{tabular}$$

Hector encodes the message by finding the degree $\leq 2$ polynomial $P(x)$ that passes through $(0,m_0)$, $(1,m_1)$, and $(2,m_2)$, and then sends you the five packets $P(0),P(1),P(2),P(3),P(4)$ over a noisy channel. The message you receive is
$$\text{DHACK} \Rightarrow 3,7,0,2,10 = r_0,r_1,r_2,r_3,r_4$$
which could have up to 1 error.

\begin{enumerate}[label=\alph*)]
\item First, let's locate the error, using an error-locating polynomial $E(x)$.  Let $Q(x) = P(x)E(x)$. Recall that
$$Q(i) = P(i)E(i) = r_i E(i), \quad \text{for} \quad 0 \leq i < n+2k.$$

What is the degree of $E(x)$? What is the degree of $Q(x)$? Using the relation above, write out the form of $E(x)$ and $Q(x)$ in terms of the unknown coefficients, and then a system of equations to find both these polynomials.
\begin{solution}
	E is a polynomial of degree 1. Q is a polynomial of degree (3+1-1=3).
	\[E(x)=(x-e_1)=x+b_0\]
	\[Q(x)=a_3x^3+a_2x^2+a_1+x^1\]
	
	I ran out of time but the rest of the problem is pretty easy. We can construct a system of equations using the $Q(i)=r_iE(i)$ property, solve the system of equations to get $Q(x)$ and $E(x)$, then we can solve $P(x)=Q(x)/E(x)$ and recompute the message.
\end{solution}

 \item Solve for $Q(x)$ and $E(x)$. Where is the error located? 


 \item Finally, what is $P(x)$? Use $P(x)$ to determine the original message that Hector wanted to send. 
 
% \textit{Hint: The message refers to a US federal agency.}
\end{enumerate}

\Question{Problem 5: Error-Detecting Codes}
Suppose Alice wants to transmit a message of $n$ symbols, so that Bob
is able to \textit{detect} rather than \textit{correct} any errors that have occurred on the way. That is, Alice wants to find an encoding so that Bob, upon receiving the code, is able to either
\begin{enumerate}[label=\roman*.)]
    \item tell that there are no errors and decode the message, or
    \item realize that the transmitted code contains at least one error, and throw away the 
        message.
\end{enumerate}
Assuming that we are guaranteed a maximum of $k$ errors, how should Alice extend
her message (i.e. by how many symbols should she extend the message, and how
should she choose these symbols)? You may assume that we work in
$\mathrm{GF}(p)$ for very large prime $p$. Show that your scheme works, and that
adding any lesser number of symbols is not good enough.

\begin{solution}
	We want to be able to send a message of length $n$ with a maximum of $k$ errors s.t. Bob is able to detect the presence of errors and decode if not. We would need to extend the original message by $k$ to send a final message of length $(n+k)$ that satisfies those requirements.
	\begin{proof}
		We claim we need to extend a message by at least $k$ to be able to detect a maximum of $k$ errors in the original message. The proof continues in two parts:
		\begin{enumerate}[label=\roman*.)]
		\item Extending the message by $k$ allows us to detect errors:
		\item Extending the message by some $k'<k$ is insufficient.
		\end{enumerate}
		For both parts of the proof, let us begin by defining the encoding from Alice.  We can denote the message as $m_1,...,m_n:m_i \in GF(q)$ where $q$ is some prime number larger than the range of $m_i$. From there, we can find the unique polynomial of degree $n-1$ s.t. $P(i)=m_i$ for $1\leq i\leq n$, i.e. it passes through all points $(i,m_i)$ for $1\leq i\leq n$. Then we set the extended portion of the message to be equivalent to $(j, P(j))$ for $n< j \leq (n+k)$.
		
		For part (i) of the proof, we have to show this encoding allows us to detect if up to $k$ errors exist in a received message $M$ (of length $n+k$). As Bob, we construct a polynomial $P'$ that fits the original message length, i.e. the first $n$ terms of $(i,m_i)$. We are then able to detect if no errors exist by validating that all points $(j, m_j)$ for $n< j \leq (n+k)$ lie in $P'$. Otherwise we can discard the message.
		
		For part (ii) of the proof, we have to show that an extension of a lesser length would not be sufficient. The logic in its simple form is that if there are $k$ errors but fewer redundant symbols, there could exist a scenario in which all redundant symbols are modified as well as a symbols in the original message, and those errors 'cancel out' in the detection algorithm. Formally, upon intercepting a message $M$, we could corrupt some symbol in $m_i:i\leq n$, construct a new polynomial $P_e(x)$ that fits this corrupted message, mutate each extended bit to fit this new polynomial, i.e., $m_{j'}=P_e(j):n < j < n+k$. This new message has $k$ corruptions, yet it would pass the detection algorithm.
	\end{proof}
\end{solution}

%\Question{Problem Error-Detecting Codes} In the realm of error-correcting codes, we usually want to recover the original message if we detect any errors, and we want to provide a guarantee of being able to do this even if there are $k$ general errors. Suppose that instead we are satisfied with detecting whether there is any error at all and do not care about the original message if we detect any errors. In class you saw that for recovering from at most $k$ general errors when transmitting a message of length $n$ you need to extend your message by $2k$ symbols and send a message of length $n+2k$. But since we don't require recovering the original message, it is conceivable that we might need less symbols.

%Formally, suppose that we have a message consisting of $n$ symbols that we want to transmit. We want to be able to detect whether there is any error if we are guaranteed that there can be at most $k$ general errors. That is, your receiver should be able to say either 'this message is completely correct' and decode it, or say 'this message has at least one error' and throw it away. How should we extend our message (i.e. by how many symbols should we extend, and how should we get those symbols) in order to be able to detect whether our message has been corrupted on its way? You may assume that we work in $GF(p)$ for a very large prime number $p$. Show that your scheme works, and that adding any lesser number of symbols is not good enough.%

% \ffsol{We claim that we need to extend our message by $k$ symbols in order to be able to detect up to $k$ errors. Suppose that the message is extended to $m_1,m_2,\dots,m_k,m_{k+1},\dots,m_{k+n}$. The encoding procedure is exactly the same as what we saw in the lecture. We do the following detection algorithm. Find the unique polynomial of degree $n-1$ that passes through points $(1,m_1)$ up to $(n,m_n)$. Call this polynomial to be $P$. If all of the points corresponding to the extended symbols $(n+1,m_{n+1})$ up to $(n+k,m_{n+k})$ lie on $P$ then we declare that there is no errors. Otherwise the message is corrupted. 

%Now we prove that if we extend the message by any number of symbols less than $k$, then the adversary can perturb the symbols such that the detector does not find it out. Suppose that the message is extended by $k-1$ symbols, $m_{n+1}$ up to $m_{n+k-1}$. Then the adversary can change symbol $m_n$ to $\tilde{m}_n$, find the polynomial that passes through the points $(1,m_1)$ up to $(n,\tilde{m}_n)$. Let this polynomial be $\tilde{P}$. Then the adversary can change the extended symbols such that all of the points $(n+1,\tilde{m}_{n+1})$ up to $(n+k-1,\tilde{m}_{n+k-1})$ lie on $\tilde{P}$. (Note that the adversary can perturb up to $k$ symbols) Therefore, the detector cannot detect that the message is corrupted. This shows that we need at least $k$ extra symbols to detect $k$ errors.
% }

\end{document}
