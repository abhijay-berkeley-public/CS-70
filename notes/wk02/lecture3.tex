\documentclass[11pt, notitlepage]{report}

	\usepackage[margin=1in]{geometry}
	\usepackage{amsmath,amsthm,amssymb,amsfonts}
	\usepackage{enumitem}
	
	\newcommand{\N}{\mathbb{N}}
	\newcommand{\Z}{\mathbb{Z}}
	\newcommand{\R}{\mathbb{R}}
	\newcommand{\A}{\alpha}
	\usepackage[parfill]{parskip}
	\usepackage{mathtools}
	\newenvironment{solution}{\paragraph{\small Solution:}}{\hfill}
	\newenvironment{theorem}{\paragraph{Theorem:}}{\hfill}
	\newenvironment{definition}{\paragraph{Definition:}}{\hfill}
	\newenvironment{problem}[2][Problem]{\begin{trivlist}
	\item[\hskip \labelsep {\bfseries #1}\hskip \labelsep {\bfseries #2.}]}{\end{trivlist}}
	
	\usepackage{pgfplots}
	\usetikzlibrary{arrows}
	\usetikzlibrary{decorations.markings}
	\usetikzlibrary{datavisualization}
	\usetikzlibrary{datavisualization.formats.functions}
	%\usepackage{pstricks-add}

	\pgfplotsset{every axis/.append style={
	                   axis x line=middle,    % put the x axis in the middle
	                   axis y line=middle,    % put the y axis in the middle
	                   axis line style={<->,color=gray}, % arrows on the axis
	                   xlabel={$x$},          % default put x on x-axis
	                   ylabel={$y$},          % default put y on y-axis
	           }}
	\pgfplotsset{compat=1.15}
	
	\newcommand{\pgraph}[4]{
		\begin{center}
		
		\begin{tikzpicture}
		\begin{axis}[
		   trig format plots=rad,
		   axis equal,
		   grid=both
		]
		\addplot [domain=#3:#4, variable=\t, samples=50, black, decoration={
		   markings,
		   mark=between positions 0.2 and 1 step 4em with {\arrow [scale=1.5]{stealth}}
		   }, postaction=decorate]
		({#1}, {#2});
		
		\end{axis}
		\end{tikzpicture}
		
		\end{center}
	}


   \newcommand{\cgraph}[3]{
	   \begin{center}
	
	   \begin{tikzpicture}
	   \begin{axis}[
	       trig format plots=rad,
	       axis equal,
	       grid=both
	   ]
	   \addplot [domain=#2:#3, variable=\x, samples=50, black, decoration={
	       markings,
	       mark=between positions 0.2 and 1 step 4em with {\arrow [scale=1.5]{stealth}}
	       }, postaction=decorate]
	   {#1};
	
	   \end{axis}
	   \end{tikzpicture}
	
	   \end{center}
	}


	
	\makeatletter
	\newcommand*{\toccontents}{\@starttoc{toc}}
	\makeatother


\begin{document}
   \title{CS 70: Lecture 3 Notes}
   \author{Abhijay Bhatnagar}
   \maketitle
   \toccontents



\setcounter{secnumdepth}{0} %% no numbering
\section{Proof By Induction}

Goal: Prove $(\forall n \in \N) P(n)$

e.g. \[
(\forall n\in\N) \sum_{i=0}^{n}i = \frac{n(n+1)}{n}
\]

\textbf{Method: }

- Prove $P(o)$ (Base case)

- For arbitrary $k >= 0$, prove $p(k) \implies p(k+1)$, therefore $(\forall n\in\N)P(n) holds)$

\textbf{Induction hypothesis:}

Assume P(k), prove P(k+1)

\begin{theorem}
\[
	(\forall n\in\N) \sum_{i=1}^{\infty}i = \frac{n(n+1)}{n}
\]

\begin{proof}
	By induction on n:
	
	Base case: $P(o): \sum_{i=0}^{n} = 0$
	
	Induction step: For arbitrary k>= 0, assume P(k): sum i=0 to k i = k(k+1)/2
	
	Prove P(k+1): 
	\[\sum_{i = 0}^{ k+1 }  = \sum_{i=o}^{k} + (k+1 )
	
	= k(k+1)/2 + (k+1)
	
	= (k+1)(k+2)/2
	
	Hence p(k \implies p(k+1) for all k>=0, hence \forall n P(n) by induction
	
\end{proof}

\end{theorem}

\begin{theorem}
	For all $n>= 3$, the sum of the interior angles of a polygon with n sides is exactly (n-2)\pi
	
	\begin{proof}
		By induction on n.
		
		Base: n= 3 \implies sum of angles = (3-2)\pi
		
		Ind. step: assume it holds for any k-gon (k-arbitrary). Prove for (k+1)-gon.
		
		Given any (k+1)-gon, cut off a triangle with a diagonal
		
		Angle sum = (angle sum of a triangle) + (angle sum of a k-gon)
				  = \pi + (k-2)\pi (using p(k)}
				  = ((k+1)-2)\pi
	\end{proof}
	
	Note: avoid common mistake of going from k-gon to k+1-gon, you should go backwards (from k+1-gon to k-gon, because that's necessary to prove the arbitrary k+1)
	

\end{theorem}

\section{Strengthening the hypothesis}

\begin{theorem}
	\[\forall n gte 1, \sum_{i=1}^{n} \frac{1}{i^2} \leq 2\]
	
	\begin{proof} By induction on n:
	
	Base: n=1:  \[\sum_{i=1}^{1} \frac{1}{i^2} == 1 \leq 2 \](True)
	
	Ind. Step: for arbitrary k, assume \[ \sum_{i=1}^{k} \frac{1}{i^2} \leq 2\]
	
	Prove same for k+1
	
	\[\sum_{i=1}^{k+1} \frac{1}{i^2} = \sum_{i=1}^{k} \frac{1}{i^2} + \frac{1}{(k+1)^2} \leq 2 + \frac{1}{(k+1)^2}\]
	
	This doesn't work.
		
	\end{proof}
\end{theorem}

\begin{theorem}

	\[\forall n gte 1, \sum_{i=1}^{n} \frac{1}{i^2} \leq 2 - 1/n\]
	
	\begin{proof} By induction on n:
	
	Base: n=1:  \[\sum_{i=1}^{1} \frac{1}{i^2} == 1 \leq 2 -1 (True)\]
	
	Ind. Step: for arbitrary k, assume \[\sum_{i=1}^{k} \frac{1}{i^2} \leq 2 - \frac{1}{k}\]
	
	Prove same for k+1
	
	\[\sum_{i=1}^{k+1} \frac{1}{i^2} = \sum_{i=1}^{k} \frac{1}{i^2} + \frac{1}{(k+1)^2} \leq 2 - \frac{1}{k} + \frac{1}{(k+1)^2}\]

	Need: \[2-\frac{1}{k}+\frac{1}{(k+1)^2} \leq 2 - \frac{1}{k+1}\]
	
	Algebraic exercise.
	
	\[-\frac{1}{k}+\frac{1}{k+1} \leq 0\]
	
	Note: You can strengthen an induction proof by trying to prove a stronger claim, because you can use that stronger assumption in your inductive step.
	
	\end{proof}
	
\end{theorem}

\begin{theorem}
	Any $2^n$ x $2^n$ chessboard can be tiled with L-shaped tiles, leaving exactly one hole adjacent to center.
	
	\textbf{Exercise}: $\forall n\geq 1, 3 | (2^{2n} - 1)$ 
	
	\begin{proof}
		Take an arbitrary $2^{k+1} $  x $ 2^{k}$ board, and chop it in a quarter, and each of those. This leaves 4 holes in each of the quarter centers. We strengthen our claim by changing the phrase "in the center" to "any desired location"
		
		Note: it is important to write "by the induction hypothesis, we can tile all four sub-boards with their desired hole positions"
	\end{proof}
\end{theorem}

\section{Strong Induction}

May as well assume all of $P(0),...,P(k)$ when proving $P(k+1)$ (consider domino visualization)

\begin{theorem}
	Every n>1 can be written as a product of primes.
	
	\begin{proof}
		Induction on n.
		
		Base: n=2: 2 is already a prime, trivial.
		
		Induction step: Assume for all integers $1<n\leq k$
		
		case i: k+1 is prime, trivially true
		case ii: k+1 is not prime, then by definition $k+1 = a\cdot b \text{for some} 1< a,b < k+1$
		
		By strong induction hypothesis, both a \& b are products of prime. 
	\end{proof}
\end{theorem}

\section{Fibonacci Numbers}

\begin{definition}
	F(0)=0 \\
	F(1)=1 \\
	F(n) = F(n-1)+F(n-2) \forall n \geq 2 \\
	0,1,1,2,3,5,8,13,21,...
\end{definition}

\begin{theorem}
	For all $n\in\N, F(n) = \frac{\phi^n-\psi^n}{\sqrt{5}}$
	
	Where: \phi = \frac{1+\sqrt{5}}{2}
	Where: \psi = \frac{1-\sqrt{5}}{2}
	
	Corollary: For large n, $F(n) \approx \frac{1}{\sqrt{5}}\phi^n=\frac{1}{\sqrt{5}}(1.618...)^n$
	
	$F(n+1)/F(n) \rightarrow \phi \text{ as n } \rightarrow \infty$
	
	Exercise: Complete proof of theorem by induction on n
	
\end{theorem}

\end{document}
